\documentclass[12pt, a4paper]{article}
\input{../templates/notes/hwsetup}
\setlength{\parindent}{1em}
\usepackage{setspace}
\title{期末考试复习提纲}
\author{于品教授放过我}
\renewcommand{\thesection}{\Alph{section}}
\renewcommand{\thesubsection}{\thesection\arabic{subsection})}
\renewcommand{\thesubsubsection}{\thesection\arabic{subsection}-\arabic{subsubsection}}
\newcommand{\normal}{\triangleleft}
\usepackage{titlesec}

\titleformat{\section}
  {\Large\bfseries\boldmath}{\thesection}{0.6em}{}

\titleformat{\subsection}
  {\large\bfseries\boldmath}{\thesubsection}{0.6em}{}

\titleformat{\subsubsection}
  {\normalsize\bfseries\boldmath}{\thesubsubsection}{0.6em}{}

\usepackage{enumitem}

\definecolor{Pampas}{HTML}{F4F3EE}
\pagecolor{Pampas}
\begin{document}
\doublespacing
\maketitle

\section{域的扩张理论基本梳理}

这一部分我们对一些重要的(我们希望研究的)扩张进行归类,并梳理他们的基本性质,我们对这一类扩张能够实施的基本操作:

\subsection{代数扩张和超越扩张}
\subsubsection{定义}
先定义代数元,若所有元都是代数元,那么扩张就是代数的,一个特例是有限扩张;

否则,扩张就是超越扩张

\subsubsection{代数扩张基本性质}
\begin{enumerate}
  \item 代数闭包的存在性:我们可以将它视作最大的代数扩张,这是对代数闭包的态射描述
  \item 传递性
  \item 代数运算封闭性
  \item 扩张的代数部分是一个域:$L^{alg}$,指的是扩张和代数闭包的交集
  \item \emph{嵌入的可延拓性},基域到闭包的嵌入一定能延拓到扩域上
\end{enumerate}

\subsubsection{代数扩张的基本操作}

\begin{enumerate}
  \item 取正规闭包,在态射意义下,我们认为正规闭包是嵌入态射$\sigma \in Hom_{K}({L,\Omega})$诱导的所有$\sigma(L)$的合成域,(因为正规扩张本身就应该视为对嵌入态射不变的扩张)
  \item 取可分闭包,在态射意义下,我们可以这样理解可分闭包:如果
  \item 需要注意的是,正规闭包往往需要"添加"元素,而可分闭包需要"减少"元素.
\end{enumerate}

\subsection{分裂域和正规扩张}
\subsubsection{分裂域的性质}
为简略期间,不再写定义

分裂域是唯一的,这是因为K嵌入本身也保持K-系数多项式

我们留意到扩张$K(\alpha) \simeq K[X]/(m_{\alpha}(X))$中未必包含所有根

假设极小多项式的另一个根$\beta \notin K(\alpha)$,那么$K(\beta) \simeq K(\alpha) \in \overline{K}$,这就是说,$K$的嵌入,延拓到$K(\alpha) $上,可以不保持$K(\alpha) $

我们自然猜测以下两个性质具有一定的关联:
\begin{center}
 域扩张包含了一个不可约多项式的所有根$\leftrightarrow$基域嵌入的延拓一定保持扩域
\end{center}

事实上,两个性质就是等价的,我们称之为正规性,据此我们定义正规扩张

\subsubsection{正规扩张的定义}
\begin{enumerate}
  \item 存在一个(一族)K-系数多项式,L是它的分裂域
  \item 任意一个(一族)K-系数多项式,它要么在L上没有根,要么在L上分裂
  \item 存在代数封闭域$\Omega$,任意给定一个K-嵌入的延拓,它都保持L
  \item 任意代数封闭域$\Omega$,任意给定一个K-嵌入的延拓,它都保持L
\end{enumerate}

\subsubsection{正规扩张的基本性质}
\begin{enumerate}
  \item 正规扩域的复合域也是正规扩域
  \item 正规扩域相对于中间域也是正规的扩域
  
  (从根的角度来看存在一族K-系数多项式,它的分裂域是L,将它们视为M-系数多项式即可,从态射的角度来看,所有M-嵌入也必须是K-嵌入)
  \item 正规扩张\emph{没有}传递性
  \item 2次扩张都是正规扩张(我们可以这样\emph{背过}这个结论:指数为2的子群都是正规子群),并且利用这个结论,很容易构造不传递的正规扩张链
  \item 给定$\sigma \in \mathbf{Gal}(L/K) $和一个元素$\alpha$的零化多项式$P(X)$,那么$\sigma(\alpha)$也被$P(X)$零化,这是定义非常自然的推论,它也非常好用
\end{enumerate}


\subsection{可分扩张}

除了下面例子以外,任何扩张都是可分的,这一节的讨论是平凡的.

\begin{example}[不可分扩张]
考虑有限域上面的有理函数域(就是多项式环的分式域)$\mathbb{F}_p(t)$,在里面添加$t^\frac1{p}$,得到的扩张是不可分扩张
\end{example}



\subsection{可分扩张的定义}
\begin{enumerate}
  \item 某个元素极小多项式无重根,就被称为可分元素,如果扩域中全体元素都是可分的,扩张就是可分的;
  \item 取定代数封闭域$\Omega$,考虑所有K-嵌入映射的延拓,如果延拓的个数就等于扩张次数,那么就是可分的.考虑特例 :单代数扩张,这个定义和上面定义1的等价性非常显然,对于有限扩张,我们可以考虑单代数扩张的链.
  \item 还有一种刻画方式,任给不同的K-嵌入,L的像也不同(也就是说不同的态射可以被区分),这可能是\emph{可分}这一称呼的来由
\end{enumerate}

我们不再过多考虑扩张是否可分,所以断言一切扩张都是可分的.

\subsection{Galois扩张}

正规且可分的扩张就是Galois扩张,所有Galois扩张都是有限的

\textbf{Remark.}实际上,无限的Galois扩张是存在的,但是据于品老师的描述,这种扩张的研究思路是naïve的:

做如下的回顾:对于无限的结构,我们往往研究与之相关的有限结构,并且定义一种极限,用有限结构的极限来逼近无限. 在这里我们使用所谓的Krull拓扑来描述收敛性

Krull拓扑的动机是这样的:

我们在Galois群$\mathbf{Gal}(L/K)$上面定义拓扑

给定一个Galois扩张$L/K$,那么$L$是所有有限中间域$M$的并集,并且$\mathbf{Gal}(L/M)$落在$\mathbf{Gal}(L/K)$的内部.因此,直观上Galois群单位元的邻域应该是$\mathbf{Gal}(L/M)$,所以我们就这样定义单位元的邻域基,再利用群的平移性质可以定义出所有元素的邻域基

对于无限的情况,我们还要修正Galois对应定理

注意到给定子群$H$,回顾课本对于Galois对应定理的证明,我们只证明了$\mathbf{Gal}(L/L^H)$是一系列包含$H$的闭集的交,所以$\mathbf{Gal}(L/L^H)$至少包含(实际上等于) $\overline{H}$,有限的情况下$\overline{H} = H$,所以Galois对应定理对于所有子群成立,然而对于无限的情况,$H$未必是一个闭集,因此我们需要修改对应定理,新的命题是:闭子群和中间域一一对应.

据于品老师说,这一部分是naïve的,而且我们不会考,所以不再赘述,并且将重点放在几种特殊的扩张中:

\subsubsection{Abel扩张}
Abel扩张指的是Galois群是交换群的扩张,它最重要的性质可能是这个
\begin{proposition}[有限生成Abel群分类定理]
  任给Abel扩张,我们都能找到一个域塔,使得$K_{i+1}/K_i$是循环扩张
\end{proposition}
它的逆命题不成立:$\mathbb{Q} - \mathbb{Q}(\sqrt{2})- \mathbb{Q}(\sqrt[4]{2})$

另一个重要的结论:添加根式得到的扩张是Abel扩张

\textbf{Remark.}于品老师最爱的例子\textbf{之一},将K上带有复乘的椭圆曲线的所有n-torsion点添加到K上面可以得到一个Abel扩张

\subsubsection{循环扩张,Hilbert 90,  Kummer理论}

我们不再叙述定义与证明,只考虑定理和相关的例子:

\begin{theorem}[Hilbert 90]
  给定n次循环扩张和Galois群生成元$\sigma$,某个元素$x$范数为1的等价条件是:不动点方程$x \cdot \sigma(y) = y$有解

  某个元素的迹为0的条件是不动点方程$x + \sigma(y) = y$有解

\end{theorem}

 \textbf{Remark.} 实际上,迹为0的元素都可以表示成不动点方程的解可以被形式化为$\mathrm{Ker}(\mathrm{Tr}) = \mathrm{Im}(\sigma -1)$,范数的命题亦然,据此我们可以发展上同调理论来推广Hilbert 90

\begin{example}
  勾股定理和Pell方程都是Hilbert 90的特例,他们分别对应扩张$\mathbb{Q}(i)/\mathbb{Q}$和$\mathbb{Q}(\sqrt{d})/\mathbb{Q}$
\end{example}

\begin{theorem}
  [Kummer]假设$K$包含了所有n次单位根,那么有限扩张$L/K$的两个命题等价:

  (i)$L/K$是n次循环扩张;

  (ii)$L$是某个$K[X]$中多项式$X^n -a$的分裂域,且$a \notin (K^{\times})^d (d|n)$ 

\end{theorem}

证明的核心是观察到$\xi^{-1}$是一个范数为1的元素,设定不动点方程的某个解为$\alpha$,那么就有$\sigma(\alpha) = \xi \alpha$


我们可以用这个定理来刻画任意一个域(未必包含所有单位根)上$X^n -a$的分裂域的Galois群

\begin{proposition}
  对于任意的域$K$,某个$K[X]$中多项式$X^n -a$的分裂域$L$一定有
  $$\mathbf{Gal}(L/K) < \mathrm{Aff}_1(\mathbb{Z}/n\mathbb{Z})$$
\end{proposition}

因为分裂域就是$K(\alpha,\xi)$,同态必须把本原单位根映射到本原单位根,将$\alpha$映射到$P(X)$的根





\section{对称多项式基本理论}
\subsection{补充:多项式基本理论} 

\subsubsection{有理系数多项式}
\begin{enumerate}
\item 本原多项式的Gauss引理
\item 整系数多项式和有理系数多项式环可约是等价的
\item 不可约判则:Eisenstein, Mod p法
\item (整系数多项式)有理根定理
\subsubsection{其他的一般多项式}
\end{enumerate}
\begin{enumerate}
\item 根的性质:代数基本定理(据此可以定义所谓的代数封闭域),Vieta定理
\end{enumerate}

\subsection{对称多项式的重要定理}
对称多项式说的是变量置换下保持不变的多项式
\begin{theorem}    [对称多项式基本定理]
    对称多项式可以表示为初等多项式的\emph{多项式}:

    用环论的语言来说,我们认为$\mathbb{Z}[X_1,X_2,\dots,X_n]^{\mathfrak{S}_n} = \mathbb{Z}[e_1,e_2,\dots,e_n]$

    证明是基于归纳法的,我们只要给出多元多项式的序(一般来说是字典序),再找到合适的方式消去最高次项,就可以归纳(递降)
\end{theorem}
\begin{theorem}[Newton]
作为一种特殊的对称多项式,幂和多项式有基于初等对称多项式的递推关系
\end{theorem}

\subsection{结式和判别式}

\subsubsection{结式的定义和等价定义}
结式是用于判别多项式是否有公共根的工具:
\begin{enumerate}

\item Sylvester矩阵定义
\item 用根的关系定义(我们需要用到分裂域)
\item 将f的根代入g定义(和第二个定义几乎一样) 
\end{enumerate}
\subsubsection{判别式的定义和等价定义}
判别式就是一个多项式所有根两两作差,平方之后得到的结果,如果无重根,那么它就非0,注意到,它也能用结式来定义(因为根的重数可以用导数来判断):

$\Delta(f) = \prod_{i<j} (\alpha_i - \alpha_j)^2 = (-1)^{n(n-1)/2} \mathrm{Res}(f, f')$
\subsubsection{结式和判别式的重要性质}

注意到$\Delta$总是被$\mathbf{Gal}(L/K)$保持,我们就知道它总是属于基域$K$
\hyperlink{3}{\hypertarget{1}}{这里是目标内容}
注意到$\delta = \prod_{i<j} (\alpha_i - \alpha_j)$是判别式的一个平方根,并且所有保持这个元素的Galois群作用恰好是偶置换,据此,我们断言

$$\delta \in K \Leftrightarrow \mathbf{Gal}(L/K) \subseteq \mathfrak{A}_n$$

据此可以计算一个简单的例子:

\begin{example}
    不可约多项式$P(X) = X^3  -2$的分裂域是$L  =\mathbb{Q}(\sqrt[3]{2},\omega)$,此时$\mathbf{Gal}(L/Q ) = \mathfrak{S}_3$

    计算的细节:显然判别式小于0所以不存在有理数平方根,因此Galois群至少包含一个对换,此外,Galois群必须是传递的,因此它逐个检验$\mathfrak{S}_3$的子群就能得到这个证明
\end{example}

除此之外,我们知道有限可分扩张都是单扩张,利用结式,我们可以计算这个单扩张的本原元的零化多项式,进一步说明这个扩张是一个多项式的分裂域,以下是一个具体的例子:

\begin{example}

  注意到$\sqrt{2}+\sqrt{3}$是$\mathbb{Q}(\sqrt{2},\sqrt{3})$的本原元素,求解它的极小多项式:

  我们需要利用这样的事实,结式为0 $\Leftrightarrow$ 两个多项式有公共根

  基于这个事实:我们假设$P(X) =X^2 -2$,$Q(X) = X^2 -3$,那么考虑结式$R(Y) = \mathrm{Res}_{X}(P(X),Q(Y-X))$

  此时$R(Y) = 0 \Leftrightarrow P(X),Q(Y-X)\text{有公共的根} \Leftrightarrow \text{Y是P的某个根和Q的某个根的和} $

  我们可以用结式的第三个定义来计算,假设$P(X)$全体根为$\{\alpha_i\}_{1 \leq i \leq n}$

  我们就知道$$R(Y) = a_m^{deg Q} prod_{i=1}^{m} Q(\alpha_i) = Q(Y -\sqrt{2})Q(Y+ \sqrt2)$$
  
  从而$R(Y) = Y^4 - 10Y^2 +1$

\end{example}

\section{Dedekind定理}
我们不加证明地给出以下用于计算分裂域Galois群中元素的定理
\begin{theorem}
  给定首一不可约的整系数多项式$P(X)$,考虑它在有理数域上的分裂域$L/\mathbb{Q}$

  如果$P(X)$在mod p意义下的不可约分解包含的次数为$(n_1,n_2,\dots,n_d)$,那么$\mathbf{Gal}(L/\mathbb{Q})$包含一个型为$(n_1,n_2,\dots,n_d)$的元素

  特别的,一个不可约多项式分裂域的Galois群中一定包含一个n-循环.
\end{theorem}

\begin{example}
  取多项式$P(X) = X^4 + 4X^3 + 2X^2 +3X -5$
  

\end{example} 
 mod2之后不可约,自然不可约,自然分裂域的Galois群包含一个4-循环

  mod3之后$\overline{P}(X) = X^4 +X^3+2X^2+1 = (X-1)(X^3-X+1)$,从而包含一个(1,3)置换

  此时,我们考虑到$\mathbf{Gal}(L/\mathbb{Q})$是一个至少12个元素的,包含4-循环的$\mathfrak{S}_4$的子群,从而它是$\mathfrak{S}_4$.

下面考虑一个更复杂的命题,它需要一个引理

\begin{lemma}
  给定一个群$G \subseteq \mathfrak{S}_n$,$G$在n元集上的作用传递,并且它包含一个对换和一个n-1循环,那么$G = \mathfrak{S}_n$

\end{lemma}

  从群论角度上来看,它非常容易证明但是看起来非常无聊.据此我们从Galois理论的角度来考察 :

  传递作用给出多项式的不可约性,n-1循环对应着多项式有modp根,对换使得Galois群不落在$\mathfrak{A}_n$中未必包含所有根

\begin{example}
  令 $K$ 为 $P(X)=X^6+22X^5+6X^4+12X^3-52X^2-14X-30$ 在 $\mathbb{Q}$ 上的分裂域,计算 $\mathbf{Gal}(K/\mathbb{Q})$ 。
\end{example}

显然这个例子中多项式不可约,而且有mod3是(1,5)的,mod5是(1,1,1,1,2)的,就得到了这个证明


\section{Kummer理论}

\begin{example}
  $P(X) = (X^5-2)(X^5 -3) \in \mathbb{Q}$的分裂域的Galois群$\mathbf{Gal}(L/\mathbb{Q})$ 
\end{example}

我们先做一个简单的情形:记中间域$K$是$X^5-2$的分裂域,它应该恰好是20次扩张(因为5次单位根对应一个4次扩张,$\sqrt[5]{2}$对应一个5次扩张)

我们考虑扩张$L/K$,我们希望说明$X^5 -3$在K上面没有根(否则由正规性$K = L$),反设根存在,那么$\mathbb{Q}(\sqrt[5]{3})$是域扩张$K/\mathbb{Q}$的一个中间域,这又对应了$\mathbf{Gal}(K/\mathbb{Q})$的一个4阶(Sylow2)子群,利用Sylow定理,这样的子群有1个或5个

更进一步,Sylow2子群对应的中间域应该是$\mathbb{Q}(\sqrt[5]{3} \cdot \xi^k) (k =0,1,2,3,4)$

此时$\mathbb{Q}(\sqrt[5]{2})$也是一个中间域,考虑到这是$\mathbb{R}$的子域,我们就知道$\mathbb{Q}(\sqrt[5]{3} =\mathbb{Q}(\sqrt[5]{2}$,但是(经过相当繁琐的计算)我们知道不可能!

因此我们知道$X^5 -3$在$K$上没有根,我们还要证明它是不可约的:

我们希望利用$X^5 -3$在$\mathbb{Q}$上的不可约性证明在K上面的不可约性:反设在k上可约,因为它没有根,可知不可约因子一个是二次多项式,一个是三次多项式

任取$\sigma \in \mathbf{Gal}(K/\mathbb{Q})$ 保持$X^5 -3$,由分解的唯一性,以及次数的的比较,
我们知道不可约因子也被$\sigma$保持,从而这说明不可约因子也是$\mathbb{Q}$的元素,这给出了矛盾!

\begin{example}
  我们可以计算$(\mathbf{Q},X^6-2)$的Galois群.
\end{example}

\section{练习题}
\renewcommand{\thesubsection}{EX.\arabic{subsection}}
\setcounter{subsection}{3}

\subsection{素谱和幂零元的关系}

\textbf{Proof.} 

先证明$\mathfrak{Nil}(A) \subseteq \displaystyle\cap_{\mathfrak{p} \in \mathrm{Spec}(A)} \mathfrak{p} $:
任取素理想$\mathfrak{p}$,$x \cdot x^{n-1} = x^n = 0 \in \mathfrak{p} \Rightarrow x \in \mathfrak{p}$

再证明$\mathfrak{Nil}(A) \supseteq \displaystyle\cap_{\mathfrak{p} \in \mathrm{Spec}(A)} \mathfrak{p} $:
反设存在$x \in \mathfrak{Nil}(A) - \displaystyle\cap_{\mathfrak{p} \in \mathrm{Spec}(A)}$ ,那么存在$\mathfrak{p}_0 $,使得$x \notin \mathfrak{p}_0$,这说明$x^n \notin \mathfrak{p}_0 $(这是因为$A/\mathfrak{p}_0 $是整环,没有零因子)

这就说明$\mathfrak{Nil}(A) - \displaystyle\cap_{\mathfrak{p} \in \mathrm{Spec}(A)} = \emptyset$,矛盾!

\textbf{Remark.}所谓Scheme理论就是从这里开始的,在过去,我们只认为多项式环中的理想能够决定一个几何对象(例如经典的:多项式$ax^2+by^2+1$生成的(素)理想定义出一个二次曲线).但是这个命题告诉我们,对任何交换环,我们都可以定义一个几何结构$\mathrm{Spec}(A)$
\subsection{}
\textbf{Proof.} 

素理想的逆像是素理想: 下面我们说明$A/\varphi^{-1}(\mathfrak{q} )$是一个整环

$\varphi$诱导了$A/\varphi^{-1}(\mathfrak{q} )$到$B/\mathfrak{q}$的一个同态$\overline{\varphi} : a + \varphi^{-1}(\mathfrak{q} ) \mapsto \varphi(a) + \mathfrak{q}$

这个映射是良定义的,而且成为一个同态

反设$A/\varphi^{-1}(\mathfrak{q} )$不是一个整环,假设$A/\varphi^{-1}(\mathfrak{q} )$有零因子$a + \varphi^{-1}(\mathfrak{q} ),b + \varphi^{-1}(\mathfrak{q} )$,那么他们被$\overline{\varphi}$映射到$\varphi(a) +\mathfrak{q}, \varphi(b) +\mathfrak{q}$

此时,不妨设$\varphi(a) \in \mathfrak{q}$,就可知$a \in \varphi^{-1}(\mathfrak{q})$,但这和$a$是零因子相矛盾!因此$A/\varphi^{-1}(\mathfrak{q} )$是一个整环,也就是说素理想的逆象也是素理想.

极大理想的逆像未必是极大理想:我们知道$\mathbb{Q}$是一个域,它的极大理想只有$\{ 0 \}$,逆像自然也是$\{ 0 \}$,但是$\mathbb{Z}$中任意非平凡的理想都包含这个逆像,所以逆像不是极大理想

\subsection{}

\textbf{Proof.} 利用归纳法来证明

归纳奠基是显然的,归纳假设命题对于$1,2,...,n$的情况都成立,递推证明$n+1$的情况:

此时,利用反证法,假设$I \subseteq \displaystyle\cup_{1 \leq i \leq n+1} \mathfrak{p}_i$且 $I \not \subseteq \mathfrak{p}_j ,\forall j$

根据归纳假设,我们知道$I \not \subseteq \displaystyle\cup_{i \neq j} \mathfrak{p}_i$,从而,我们可以取出元素$a_i$,满足$a_i \notin \mathfrak{p}_j (\forall i \neq j)$,此时,取$a = a_{n+1} +a_1a_2\cdots a_n$,那么$a \notin \mathfrak{p}_{k} ,\forall k \in {1,2,\dots,n+1}$,就导出了矛盾.

\textbf{Remark.}我们有以下几何直观:

\subsection{}

\textbf{Proof.} 由题设,可以取出$a \in I,b \in J$ 使得$a+b =1$,那么,考虑$(a+b)^{2n-1}$二项式展开,并考虑里面a次数出现不少于n次的单项式$w = b^{m_0} \cdot (a \cdot b^{m_1}) \cdot (a \cdot b^{m_2}) \cdots (a \cdot b^{m_l}) (l>n)$,这个单项式的每一项都是$I$的元素,因此乘积是$I^n$的元素.因此,所有这样的单项式的和也是$I^n$的元素

反过来,剩下所有的单项式的和是$J^n$的元素,从而我们就构造出了一对和为$1$元素,他们分别属于$I^n,J^n$

\subsection{}

由题意$K(\alpha)/K,K(\beta)/K$分别是$p =deg P,q = deg Q$次的扩张(其中$p,q$互素),那么他们均为$K(\alpha,\beta)/K$的中间域,从而$[K(\alpha,\beta):K] >= p\cdot q$,从而$[K(\alpha,\beta):K(\beta)] >=  p$,又因为$P(X)$给出了$K(\beta)[X]$之中$\alpha$的一个$p$次零化多项式,从而极小多项式就是$P(X)$

据此,次数应该是6

\subsection{}

下面记$\xi \coloneq e^{\frac{2 \pi i}{n}}$,那么记$\alpha = \frac{\xi + \xi^{-1}}{2}$,待求扩张就是$\mathbb{Q}(\alpha)/\mathbb{Q}$

考虑$\mathbb{Q}(\alpha,\beta)/\mathbb{Q}(\alpha)$,其中$\beta =\frac{\xi - \xi^{-1}}{2}$

这是一个二次扩张,因为$\beta^2+(1-\alpha^2) = 0$,并且$\mathbb{Q}(\alpha,\beta)/\mathbb{Q}$就是分圆扩张,其次数是$p-1$,从而$\mathbb{Q}(\alpha)/\mathbb{Q}$的次数是$\frac{p-1}2$

\hyperlink{4}{\hypertarget{2}{\subsection{}}}

任取$\sigma \in\mathbf{Gal}(\mathbb{Q}(\sqrt{2},\sqrt{3},\sqrt{5})/\mathbb{Q})$,我们知道$\sigma(\sqrt{2})^2 = 2,\sigma(\sqrt{3})^2 = 3,\sigma(\sqrt{5})^2 =5$,从而$\sigma(\sqrt{2}) = \pm \sqrt{2},\sigma(\sqrt{3}) = \pm \sqrt{3},\sigma(\sqrt{5}) = \pm \sqrt{5}$,这说明Galois群是$(\mathbb{Z}/2\mathbb{Z})^3$。

用归纳的方法方法说明$\mathbb{Q}(\sqrt{p_1}+\sqrt{p_2}+\dots+\sqrt{p_d})/\mathbb{Q}$(其中$p_i(i =1,2,\ldots,d)$是互不相等的素数)是一个$2^d$次的扩张。

考虑域扩张$\mathbb{Q}(\sqrt{p_1}+\sqrt{p_2}+\dots+\sqrt{p_{d-1}},\sqrt{p_d})/\mathbb{Q}(\sqrt{p_1}+\sqrt{p_2}+\dots+\sqrt{p_{d-1}})$,只要证明这个扩张是2次的。

将$\mathbb{R}$视为$\mathbb{Q}$-线性空间,$\{1,\sqrt{p_1},\sqrt{p_2},\dots,\sqrt{p_{d}},\dots,\sqrt{p_ip_j},\dots,\sqrt{p_1p_2\dots p_d}\}$是线性无关的,据此$\sqrt{p_d} \notin \mathbb{Q}(\sqrt{p_1}+\sqrt{p_2}+\dots+\sqrt{p_{d-1}})$,从而极小多项式次数至少为2,又因为$X^2 -p_d$就是一个2次零化多项式,就可知极小多项式就是$X^2 -p_d$,递降就可以得到$\mathbb{Q}(\sqrt{p_1}+\sqrt{p_2}+\dots+\sqrt{p_d})/\mathbb{Q}$是一个$2^d$次扩张。 

结合包含关系就可以得到$\mathbb{Q}(\sqrt{2}+\sqrt{3}+\sqrt{5}) = \mathbb{Q}(\sqrt{2},\sqrt{3},\sqrt{5})$。


\subsection{}

\textbf{Proof.} 由题意$K/\mathbb{Q}$是一个正规扩张,$\mathbb{C}$是一个代数封闭域.

将复共轭映射限制在$K$上,这是$K$的一个保持$\mathbb{Q}$的嵌入,由正规扩张的定义,这个嵌入也保持$K$,这也就是说复共轭映射保持$K$

\subsection{}

用Galois对应定理重新描述这个命题:

原命题等价于"给定$H\leq G$,那么$\displaystyle{\cap_{g \in G}}  gHg^{-1}$是$G$的正规子群,而且任意$N<H,N\triangleleft G$都满足$N \leq \displaystyle{\cap_{g \in G}}  gHg^{-1}$"

任取$N \leq H, N \triangleleft G$,任取$g$,$N$满足$g^{-1}Ng \subseteq H$,也就是$N \leq gHg^{-1}$,从而$N$落在$H$的正规核中

再证明正规核是正规子群,据定义,正规核中的元素在共轭映射下仍然落在正规核中,所以正规核是正规子群

\subsection{}

\textbf{Proof.} 为了利用Galois对应定理,我们只需要说明$L/M_0$是一个Galois扩张:这是因为$L/K$是正规且可分的,自然分别给出了$L/M_0$的正规性和可分性.

由Galois对应定理,$H \triangleleft N_G(H)$,从而它对应的扩张$M/M_0$是正规扩张.

进一步,如果$M/M'$是正规扩张,那么$M'$一定对应了$G$中的一个子群,它正规化$H$,我们知道$N_G(H)$包含了所有正规化$H$的$G$的子群,据此$M' \supseteq M_0$.

\subsection{}
见\hypertarget{3}{\hyperlink{1}{B3.3判别式的性质}}

进一步,因为Galois群在根集的作用是传递的,如果$\mathrm{Disc}(P)$是完全平方数,那么Galois群落在交错群中,交错群能够传递作用在$\{1,2,3\}$的子群只有它本身,因此Galois群只能是交错群.

如果$\mathrm{Disc}(P)$不是完全平方数,那么它至少包含一个奇置换(也就是对换),因为它是传递的子群,所以它只能是对称群.

\subsection{}

假如有一个$P(X)$满足$deg P>2$,考虑$P(X)$的分裂域$M$,那么$M$对应了一个指数为$degP$的,$\mathfrak{S}_n$的子群,考虑到任意$\mathfrak{S}_n,(n \geq 5)$子群要么是交错群,要么指数至少为$n$,就说明了$degP$至少为n.

为了找到$n =4 $情况下的反例,我们只需要考虑$\mathfrak{S}_4$的8阶子群(之一)$\langle(13),(1234)\rangle$

例如:我们假设$X^4-X-1$在$\mathbb{Q}$上的分裂域,并假设四个根为$\alpha_1,\alpha_2,\alpha_3,\alpha_4$,

那么我们可以构造出元素:$$\theta_1 = \alpha_1\alpha_2+\alpha_3\alpha_4,\theta_2 = \alpha_1\alpha_3+\alpha_2\alpha_4,\theta_3 = \alpha_1\alpha_4+\alpha_2\alpha_3$$

其中$\theta_2$是被$D_4$保持的元素,而$\theta_1,\theta_3$则是$\theta_2$在Galois群中映射下的像:

下面我们计算$(X-\theta_1)(X-\theta_2)(X-\theta_3)$

其中$X^4-X-1$的二次项为0说明$\theta_1+\theta_2+\theta_3 = 0$

$\theta_1\theta_2+\theta_2\theta_3+\theta_1\theta_3 = (\alpha_1+\alpha_2+\alpha_3+\alpha_4)(\alpha_1\alpha_2\alpha_3+\alpha_2\alpha_3\alpha_4+\alpha_1\alpha_2\alpha4+\alpha_1\alpha_3\alpha_4)-4\alpha_1\alpha_2\alpha_3\alpha_4 =-4$

$\theta_1\theta_2\theta_3 = (\alpha_1+\alpha_2+\alpha_3+\alpha_4)(\alpha_1^2+\alpha_2^2+\alpha_3^2+\alpha_4^2) +\sum_{i<j<k}(\alpha_i\alpha_j\alpha_k)^2$

从而这个极小多项式是$X^3 +4X -1$


\subsection{}

考虑域塔$L - K(x) -K$,和对应的Galois群子群链${\mathbf{1}} < H \leq \mathbf{Gal}(L/K) $

交换群的子群都是正规的,$H$也是正规的,对应地$K(x)/K$是正规扩张.

此时$P(X) \in K[X]$在$K(x)$上有一个根,据正规扩张的性质,$K(x)$是$P(X)$的分裂域,也就是$L = K(x)$

\subsection{}

为证明它是Galois扩张,只需要证明:

1.扩张是可分的 :这是因为$\mathbb{Q}$特征为0;

2.扩张是正规的 :显然,这个域是一族$\mathbb{Q}[X]$中多项式$\{P_i(X) = X^2 -p_i \}_{i \in {1,2,3,\dots,d}}$在$\mathbb{Q}$上的分裂域,从而它是正规的.

\hypertarget{4}{\hyperlink{2}{Ex.10}}中已经证明了它是一个$2^d$次扩张,并且考虑到形如$\sigma: \sqrt{p_i} \mapsto \varepsilon_i \sqrt{p_i} (\varepsilon_i \in \{\pm 1\})$的元素已经给出了$\mathbf{Gal}(L/\mathbb{Q})$中两两不等的$2^d$个元素,因此直接观察群元素的性质可知

$$\mathbf{Gal}(L/\mathbb{Q}) \simeq( \mathbb{Z}/2\mathbb{Z})^d$$

如果$\sqrt{15} \in \mathbb{Q}(\sqrt{10},\sqrt{42})$

那么我们有如下的域塔:

(i) $\mathbb{Q} - \mathbb{Q}(\sqrt{15}) - \mathbb{Q}(\sqrt{10},\sqrt{42})$

(ii) $\mathbb{Q} - \mathbb{Q}(\sqrt{10}) - \mathbb{Q}(\sqrt{10},\sqrt{42})$

(iii)$\mathbb{Q} - \mathbb{Q}(\sqrt{42}) - \mathbb{Q}(\sqrt{10},\sqrt{42})$

(iv) $\mathbb{Q} - \mathbb{Q}(\sqrt{105}) - \mathbb{Q}(\sqrt{10},\sqrt{42})$

并且这些域塔中相邻的域扩张都是2次的,我们知道4阶群的2阶子群至多3个,而且(ii)(iii)(iv)给出的域塔是两两不同的,所以(i)必须等于这三者之中的某一个.

显然$\sqrt{15} \notin \mathbb{Q}(\sqrt{10}),\mathbb{Q}(\sqrt{42}),\mathbb{Q}(\sqrt{105})$,就给出了证明.


\renewcommand{\thesubsection}{\thesection\arabic{subsection})}

\section{没有这一节}
F不存在.


\end{document} 