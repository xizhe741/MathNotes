\documentclass[12pt, a4paper]{article}
\input{../templates/notes/note-setup}
\usepackage{parskip}
\usepackage{setspace}
\title{Midterm Rivision for Analysis III}
\author{Xizhe Feng}
\definecolor{Pampas}{HTML}{F4F3EE}
\pagecolor{Pampas}
\begin{document}

\maketitle
\tableofcontents

\subsection{数项级数}

\subsection{正项级数的收敛判别法}
正向级数收敛判别法中非平凡的思想只有比较判别法:
\begin{enumerate}
    \item 部分和有界$ \Leftrightarrow $收敛
    \item 级数收敛 $ \Rightarrow $ 数列收敛 
    \item 比较判别法 
    \item D'Alambert比值判别法
    \item Cauchy根式判别法:用的是上极限
    \item 积分判别法 必须使用递减的函数
    \item Raabe 判别法
    \item Cauchy凝聚判别法 常用于控制带有对数的级数
    \item Gauss判别法 
    \item Betrand判别法:这个判别法是我们课上讲到的Gauss 判别法,它是对上面判别法在$\lambda =1$时候的分类。
    \item Kummer判别法:这是说如果存在$b_n$,使得$\sum\limits_{n=1}^{\infty} \frac1{b_n}$发散,且充分大的n,能够使得$a_nb_n-a_{n+1}b_{n+1}$有正下界,那么就收敛,否则发散。可以认为前面的Raabe,Gauss,Betrand判别法都是它的特例
    \item Abel-Dini定理用于判别:定理将在下面叙述。
\end{enumerate}
\emph{前面使用极限的判别法之中,即使极限不存在,也可以利用上下极限进行判别!}

我们还有如下定理:
\begin{theorem}
    [Abel-Dini] 如果一个级数$\displaystyle\sum\limits_{n=1}^{\infty} a_n$发散,那么$\displaystyle\sum\limits_{n=1}^{\infty} \frac{a_n}{S_n}$发散,且$\displaystyle\sum\limits_{n=1}^{\infty} \frac{a_n}{S_nS_{n-1}^\varepsilon}$收敛。进一步,如果$\frac{a_n}{S_n}$趋向于0,那么$\displaystyle\sum\limits_{n=1}^{N}\frac{a_n}{S_n} \sim ln(S_N)$
\end{theorem}
我们不叙述这个定理的证明(虽然它是初等的),这个定理揭示了如下事实:给定任意发散的级数,我们能够构造出一个发散得更慢的级数。对于收敛的级数,我们可以将上面定理中的部分和 $S_N$ 替换为余项 $r_N$,据此构造出一个收敛得更慢的级数。

Abel-Dini定理告诉我们一个事实:不存在一个完美的,可以用于一切比较判别法的级数。这也是我们现在有如此多收敛性判别法的原因。除此之外,Abel-Dini定理还给出了一些级数收敛性的事实(例如1/(nlnn),我们知道它是1/n及其部分和直接构造出的发散最慢的级数)

还有一些其他的技巧,例如素数的倒数和问题,这些技巧有待补充。

\subsection{一般数项级数的收敛判别法}
我们有两种方式判别:一是利用正向级数相关的结论:绝对收敛$ \Rightarrow $收敛(这是完备空间的性质),一是利用Abel判则。我们声称,这一部分真正非平凡的思想只有Abel求和公式:
$$\displaystyle\sum\limits_{n=1}^{N} (a_nb_n) = S_Nb_N - \displaystyle\sum\limits_{n=1}^{N-1} S_n(b_{n+1}-b_n)$$
这个公式有如下两个推论,他们的证明是显然的。
\begin{enumerate}
    \item Dirichlet判别法:如果 $S_n$有界且 $b_n$单调收敛到0,那么$\displaystyle\sum\limits_{n=1}^{N} (a_nb_n)$收敛
    \item Abel判别法:如果$S_n$收敛,$b_n$单调有界,那么$\displaystyle\sum\limits_{n=1}^{N} (a_nb_n)$收敛
\end{enumerate}

\subsubsection{条件收敛}

我们注意到,存在一些级数,他们收敛但不是绝对收敛。我们引入一种收敛性质称为条件收敛。关于条件收敛的级数,我们有如下的定理:

\begin{theorem}[Riemann重排定理]
    证明非常类似于Hilbert的旅馆:我们可以将级数分为正部和负部,我们知道两部分都必然收敛到0(这是Cauchy收敛原理给出的),假设我们希望重排得到的目标是L,只需要适当地交替取正负部元素,就可以使得部分和的一个子列收敛到L。又因为部分和与L的差值被子列中相邻的两部分控制,因此部分和收敛到L。
\end{theorem}

还有一个推论,其证明方法是完全类似的:

\begin{corollary}
    一个递减的正数序列${a_n}$如果发散,那么可以适当添加正负号使得新的数列收敛到任意实数。
\end{corollary}

为了证明,我们只要依次决定每一个 $a_i$的正负号就可以了。

Riemann重排定理还有如下的一些推广:有限维空间中,能够证明级数收敛到一个仿射子空间中,无穷维空间中,重排能够收敛到的子集往往没有这么好的结构。

\begin{theorem}
    [Lévy-Steinitz 定理]$\mathbb{R}^n$中收敛但不是绝对收敛的级数可以重排使得新的序列收敛到一个仿射子空间的任意元素。
\end{theorem}

\subsubsection{数项级数的计算方法}

\begin{enumerate}
    \item 裂项:注意一些特别的裂项 $arctan(\frac{1}{n^2+n+1}) =arctan(n+1)-arctan(n)$
\end{enumerate}

\subsubsection{Cauchy积}


\subsubsection{乘积级数}

\subsection{函数的级数理论}

\subsubsection{函数项级数的判别法}
\begin{enumerate}
    \item Weierstrass M-test,如果级数每一项上确界$M_n$收敛那么级数绝对一致收敛(或者称正规收敛,这只是名称上的区别)
    \item Dini定理:如果连续函数单调地收敛到一个连续函数,那么它是一致收敛的(利用紧区间套原理或者开覆盖原理可以证明)
    \item Dirichlet判别法:如果函数$a_n(x)$部分和一致有界,且$b_n(x)$单调一致收敛至0,那么$a_n(x)b_n(x)$的部分和一致收敛。
    \item Abel判别法:如果函数$a_n(x)$部分和一致收敛,且$b_n(x)$单调一致有界,那么$a_n(x)b_n(x)$的部分和一致收敛。
    \item 比值判别法:利用比值的一致上极限。
    \item 根值判别法:利用根值的一致上极限。
\end{enumerate}
\begin{example}
    一个级数$\displaystyle\sum\limits_{n=1}^{\infty} a_n$收敛(未必绝对收敛),那么幂级数$\displaystyle\sum\limits_{n=1}^{\infty} a_nx^n$在R<1的[-R,R]上是一致收敛的。
\end{example}
前面的例子之中,我们知道了幂级数$f(x) \coloneq \displaystyle\sum\limits_{n=1}^{\infty}a_nx^n$在(-1,1]上是处处收敛的,我们希望收敛到的函数$f(x)$是在1处是左连续的,这需要再一次利用Abel求和公式与截断方法,证明留作练习。

\subsubsection{函数项级数的计算方法}

Abel方法:
\begin{example}
    求解关于自变量r的幂级数$ S(r) = \displaystyle\sum\limits_{n=1}^{\infty} \frac{sinnx}{n} \cdot r^n$

    解:先在(-1,1)上求导得到 $S'(r) = \displaystyle\sum\limits_{n=1}^{\infty} sinnx \cdot r^{n-1}$,利用复数,我们可以得到$S'(r) = Im(\frac{e^{ix}}{1-re^{ix}}) = \displaystyle \frac{\sin x}{1-2r\cos x+r^{2}}$对$S'(r)$(关于r)积分就可以得到积分结果为$arctan(\frac{r-cosx}{x})$,从而$S(x) = \frac{\pi-x}2$,利用左连续性就可以得到结果。
\end{example}

\end{document}