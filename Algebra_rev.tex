\documentclass[12pt, a4paper]{article}
\input{../templates/notes/hwsetup}
\setlength{\parindent}{1em}
\usepackage{setspace}
\title{Group and Galois theory - Midterm Rivision}
\author{冯熙喆}
\renewcommand{\thesection}{\Alph{section}}
\renewcommand{\thesubsection}{\thesection\arabic{subsection})}
\renewcommand{\thesubsubsection}{\arabic{subsubsection}.}
\newcommand{\normal}{\triangleleft}
\usepackage{titlesec}

\titleformat{\section}
  {\Large\bfseries\boldmath}{\thesection}{0.6em}{}

\titleformat{\subsection}
  {\large\bfseries\boldmath}{\thesubsection}{0.6em}{}

\titleformat{\subsubsection}
  {\normalsize\bfseries\boldmath}{\thesubsubsection}{0.6em}{}

\usepackage{enumitem}

\definecolor{Pampas}{HTML}{F4F3EE}
\pagecolor{Pampas}
\begin{document}
\doublespacing 
\maketitle

\subsection*{Tips} 在相关习题上遇到困难的时候,最好的做法是检查自己是否已经实现了这个群,而不是直接查看答案。

\subsubsection{}
就是所有阶整除$n$的循环群。

\subsubsection{}
假设$G$不是有限群。由题意,它的循环子群是有限的,假设我们已经取出了所有两两不等的循环子群$\{\langle g_1\rangle,\langle g_2\rangle\cdots\langle g_r\rangle\}$,我们知道这些循环子群都是有限群(否则存在一个循环子群有无限个循环子群),因为这些有限循环子群个数是有限的,总存在$g_{r+1} \in G$不属于上述的任何一个子群。显然它能够生成一个循环子群。这个子群不是上述任意循环子群之一。这便导出了矛盾。

\textbf{Remark:}这个命题对可数的情形也是成立的,但是逆命题对于可数的情形不成立(一个阶数可数的群可能有不可数个子群)。

\subsubsection{}

Ker $ = Z(G)$的证明是平凡的,这就是中心的定义。

Im 的正规性可以这样验证:任给$g$,$\phi Int(g) \phi^{-1} = Int(\phi(g))$。

\subsubsection{}

任取$g,n$,$(gng^{-1}n^{-1}) \in \text{Ker} \phi \Rightarrow (gng^{-1}) \in N$。

\subsubsection{}

任意给定$n \in N \cap K,k \in K$,$knk^{-1}$仍然属于$N$,也仍然属于$K$。

考虑自然的同态$K \to KN/N$,$k \mapsto kN$,它的核自然就是$K \cap N$。任取$KN/N$中的元素,它都形如$k_iN$,所以它显然成为某个元素的像,因此这个同态是满射。

据同态第一定理就得到了证明。

\subsubsection{}

$\Rightarrow$利用$(hk)^{-1} = k^{-1}h^{-1} \in KH$可得 $HK \subset KH$,$kh =(h^{-1}k^{-1})^{-1} \in HK$,那么$KH \subset HK$,得证。

$\Leftarrow$只要验证封闭性,这是显然的。

\subsubsection{}

$H \cap K$ 是$H$的子群,考虑左陪集$\{h_1(H \cap K),h_2(H \cap K)\dots,h_n(H \cap K)\}$,集合的大小自然是$H/(H \cap K)$,
不难证明$h_1K,h_2K,\dots,h_nK$的两两不交且并成了$HK$。从而命题的计数成立。

\subsubsection{}
显然$H \cap K < H$。

由(7)的论证我们知道,如果$h_i(H \cap K) \neq h_j(H \cap K)$,那么$h_iK \neq h_jK$,因此如果$\Lambda$ 是一个指标集,使得$\{h_\lambda\}_{\lambda \in \Lambda}$,那么$\{h_\lambda K\}_{\lambda \in \Lambda}$一定是$G$中两两不等的左陪集。因此$[H : H \cap K] ⩽ [G : K]$。

指标有限的情况下,如果等号成立,$\Lambda$ 的大小就是$[G : K]$,可以知道$\{h_\lambda K\}_{\lambda \in \Lambda}$就是$G$关于$K$的一个陪集分解,因此$G = HK$,又因为$HK$已经构成一个群,就可以知道$HK = KH$,因此命题正向成立。

下面说明命题逆向成立:如果$G=KH$,那么就有$HK = KH = G$,它们都是群,$\{h_\lambda K\}_{\lambda \in \Lambda}$是$HK$的一个陪集分解,自然它也是$G$的一个陪集分解。同时
$\{h_\lambda H \cap K\}_{\lambda \in \Lambda}$又给出了$H$的陪集分解,从而两个指标相等(均为$\Lambda$ 的大小)。

\subsubsection{}

由(7),(8)可知$H \cap K$是$H$的有限指标子群。显然如果$\{g_\alpha H\}_{\alpha \in A}$是$H$在$G$中的左陪集,$\{h_\beta (H \cap K)\}_{\beta \in B}$是$H \cap K$在$H$中的左陪集,那么
\[
\bigcup_{\alpha \in A,\beta \in B} g_\alpha h_\beta (H \cap K) = G,
\]
因此$(H \cap K)$在$G$中的左陪集包含于$\{g_\alpha h_\beta (H \cap K)\}_{\alpha \in A,\beta \in B}$之中,进而我们就得到了$[G : H \cap K] \leq [G : H][H : H \cap K] \leq [G : H][G : K]$。


如果$G \neq KH$,那么第二个不等号不能取等,因此只需要验证 $G =KH$ 的情况能够取等。此时,我们可以改写$H$在$G$中的左陪集为$\{k_\alpha H\}_{\alpha \in A}$($k_\alpha \in K$),完全类似地,我们可以得出,如果$k_1h_1 H \cap K = k_2h_2 H \cap K$,那么$k_1k_2^{-1},h_1h_2^{-1} \in H \cap K$,因此确实能够取等。

\subsubsection{}

假设$\mathfrak{S}_n$有一个非平凡正规子群$N$,那么$\mathfrak{A}_n \cap N$也是一个正规子群。

如果$N$之中的偶置换有且仅有$\{\mathbf{1}_{\mathfrak{S}_n}\}$,那么它里面的任意两个非单位元素的乘积是一个偶置换,因此它们互为逆元,进一步,这个群至多$2$个元素,这样的非平凡正规子群不存在。因此$N$之中必有非平凡的偶置换.

利用$N$中有非$\mathbf{1}_{\mathfrak{S}_n}$偶置换这一性质,$\mathfrak{A}_n \cap N$是$\mathfrak{A}_n$的一个非平凡正规子群,据$\mathfrak{A}_n$的单性可以得出$\mathfrak{A}_n \cap N = \mathfrak{A}_n$,且$\mathfrak{S}_n$没有比$\mathfrak{A}_n$更大的子群,因此$N$只能是$\mathfrak{A}_n$。

\subsubsection{}
先证明这确实是一个同态,乘法交换律给出了

$\displaystyle\prod_{1 \leq i<j\leq n} \frac{\tau(i)-\tau(j)}{i-j} \frac{\sigma(i)-\sigma(j)}{i-j} = \prod_{1 \leq i<j\leq n} \frac{\tau(\sigma(i))-\tau(\sigma(j))}{\sigma(i)-\sigma(j)} \frac{\sigma(i)-\sigma(j)}{i-j}$

因此确实是同态。再证明在生成元上面符号映射是一致的,所以它就是之前的符号映射。

\subsubsection{}

利用陪集分解将$G$中的元素表示为$g^k z$(表示不必唯一),就可以得到交换性。

\subsubsection{}

$\Rightarrow$ 这由双传递性的定义保证。

$\Leftarrow$ 若这样的集合存在,那么我们可以将$(g,g') \notin \Delta$映射到任意$(g,g'') \notin \Delta$。考虑到Stab$(x'')$也传递地作用于$X-\{x''\}$,可以将它映射到任意$X \times X - \Delta$中的元素$(g''',g'')$,这就满足了双传递性的定义。

\subsubsection{Jordan的定理}

考虑对集合$\{(g,x)\mid gx =x\}$进行计数,如果原命题不成立,那么每一个$g_i$至少带来集合中的一个元素,且$1_G$,因此集合的大小大于$|G|$。然而对$x$计数,我们知道$|\text{Stab}(x)| = |G|/|X|$,且$\text{Stab}(x)$两两共轭,大小相等,可以得出$|S| = |G|$。这导出了矛盾。

\subsubsection{Ore的定理}

我们利用前面的命题来证明:在左诱导表示里面,$H$包含了表示的核$\text{Ker} \tau \subseteq H$,据此我们希望证明:$H$就是$\text{Ker} \tau$:

因为$|Im \tau| = |G:\text{Ker} \tau| = |G:H||H:\text{Ker} \tau| =  p|H:\text{Ker} \tau|$,我们得到$Im \tau$之中包含素因子$p$,因此$Im \tau$作为$\mathfrak{S}_p$的子群只能是$p$阶群(否则$|G|$有更小的素因子)。这就是说$|H:\text{Ker} \tau| =1$,即$H = \text{Ker} \tau \triangleleft G$。

\subsubsection{}

我们知道$GL(2;\mathbb{F}_p)$中有$(p^2-1)(p^2-p)$个元素,从而它的Sylow-$p$子群的阶数是$p$(它是一个循环群),个数应该整除$(p^2-1)(p^2-p)$且模$p$余$1$。那么Sylow $p$-子群的个数可能是$1,p+1,np+1 (n \geq 2)$。下面证明个数只能是$p+1$。
这个子群作为一般线性群的子群,Sylow $p$-子群至少包括如下两个:

$\mathit{U}_1=\{g\mid g\text{是对角元素为}1\text{的上三角矩阵}\}$,

$\mathit{L}_1=\{g\mid g\text{是对角元素为}1\text{的下三角矩阵}\}$,因此Sylow-$p$子群至少有两个。

下面我们证明这样的Sylow $p$-子群的个数不多于$p+1$个:任给Sylow $p$-子群$H = \langle h\rangle$,那么$h$在$\mathbb{F}_p[X]$之中适合多项式方程$X^p-1=0$,这个方程同时等价于$(X-1)^p=0$,因此作为线性算子的$h$必然只有特征值$1$,且特征值$1$的特征子空间一定是一个线性真子空间(那么这个空间应当是$1$维的)。下面我们建立Sylow $p$-子群之集$S$到$\mathbb{PF}_p^2$之间的映射
$$f:S \to \mathbb{PF}_p^2,\ \langle h\rangle \mapsto \text{Ker}(h-1)$$
下面证明这个映射是Well-defined的,我们需要说明任意$\langle h\rangle$以及$h_1,h_2 \in \langle h\rangle$,总有$\text{Ker}(h_1-1) = \text{Ker}(h_2-1)$,这是显然的,因为两个元素互为对方的幂次。

再证明这个映射是一个单射:如果$h_1,h_2$有着相同的特征空间,那么他们属于同一个群。

我们只需要验证$\text{Stab}(\mathrm{span}\{e_1\})$的大小:
这样的矩阵一定形如
$\scalebox{0.6}{$\begin{pmatrix} 1 & b \\ 0 & d \end{pmatrix}$}$,
它的$p$次方的对角元素是$1$和$d^p = 1$(注意要区分群乘法和矩阵的乘法),
利用数论的知识我们知道$d$只能为$1$,
这就是说明元素都形如
$\scalebox{0.6}{$\begin{pmatrix} 1 & b \\ 0 & 1 \end{pmatrix}$}$,
也就是说这个稳定子至多$p$个元素,就说明了有着相同特征空间的$h_1,h_2$属于同一个群,也就证明了这是一个单射。


因此Sylow $p$-子群的个数不多于$|\mathbb{PF}_p^2|$,也就是说Sylow $p$-子群的个数至多$p+1$,进一步,它的个数就是$p+1$。

\subsubsection{}

考虑到这个Sylow $p$-子群$H$自然地作用于集合$\{1,2,3,\dots, n\}$,我们知道任意元素轨道的长度要么是$1$,要么是$p$。显然至少有一个长度为$p$的轨道,假设长度为$p$的轨道一共有$k$个,记作$\{\Omega_1,\Omega_2 ,\dots ,\Omega_k\}$。

自然地,我们有诱导表示$\rho_i: H \to \mathfrak{S}_{\Omega_i}\simeq \mathfrak{S}_{p}$,进一步我们有表示
$$\rho: H \to \mathfrak{S}_{p} \times \mathfrak{S}_{p} \times \dots\times \mathfrak{S}_{p},\ h \mapsto \big(\rho_1(h),\rho_2(h),\dots,\rho_k(h)\big)$$

我们希望说明$H$被实现为交换群$\mathfrak{S}_{p} \times \mathfrak{S}_{p} \times \dots\times \mathfrak{S}_{p}$的子群,这等价于证明$\rho$是一个单射:这是显然的,如果某个$h$在每一个轨道上都是恒等映射,那么它只能是群的单位元素,因此$\rho$确实是一个单射。

$H$作为交换群的子群自然是交换的。

\subsubsection{}

类似上题,我们考虑Sylow $p$-子群$H$自然地作用于集合$\{1,2,3,\dots ,p^2\}$。

先证明作用是传递的(等价于说明轨道长度是$p^2$):若不然,集合由若干个大小为$p$的轨道和若干个大小为$1$的轨道组成。如果大小为$1$的轨道数目不为$0$,那么至少有$p$个,这种情况下$H$可以被实现为一个对称群$\mathfrak{S}_{p(p-1)}$的子群。对$p$的幂次进行计数,我们知道$H$不可能是$\mathfrak{S}_{p(p-1)}$的一个子群。因此集合由$p$个$p$阶轨道组成。此时,我们可以构造类似上题的诱导表示:
$$\rho: H \to \underbrace{\mathfrak{S}_p \times \mathfrak{S}_p \times \dots \times \mathfrak{S}_p}_{p \text{个} \mathfrak{S}_p}$$
然而,再一次对$p$的幂次进行计数,我们知道$H$不可能是$\underbrace{\mathfrak{S}_p \times \mathfrak{S}_p \times \dots \times \mathfrak{S}_p}_{p \text{个} \mathfrak{S}_p}$的子群。因此集合的作用是传递的。

由Jordan的定理,我们知道至少存在一个$p^2$-循环$g$,显然,$\mathfrak{S}_{p^2}$之中和 $g$交换的元素只有$\{\mathbf{1}_{\mathfrak{S}_{p^2}},g,g^2,\dots,g^{p^2-1}\}$,因此$H$里面包含不和$g$交换的元素。

\subsubsection{}

考虑$H$在 $\mathfrak{S}_n$的左诱导表示。作用是传递的$\Rightarrow$表示同态的$\text{Ker}$不是$\mathfrak{A}_n$或$\mathfrak{S}_n$,因此$\text{Ker}$是$\{\mathbf{1}_{\mathfrak{S}_{n}}\}$,同时$H$被嵌入到了陪集$\mathbf{1}_{\mathfrak{S}_{n}}\cdot H$的稳定子群之中。这个稳定子群同构于$\mathfrak{S}_{n-1}$的一个子群。再考虑阶数就知道$H\simeq\mathfrak{S}_{n-1}$。

\subsubsection{}

假设阶数为$p^{m} (m < k)$的情形已经得到了证明。同时我们假设$G$是非交换群(交换群的情况,由有限生成Abel群的分类,我们知道它是$\mathbb{Z}_p^k$,此时命题显然)。

$Z(G)$非平凡(这是类方程的自然结论)且$Z(G)$是$G$的真子群:

(1) 如果$p^l \leq Z(G)$,那么由归纳假设,$Z(G)$中有一个阶为$p^l$的子群,这个子群是中心的一部分$\Rightarrow$它是正规的。

(2) 如果$p^l > Z(G)$,假设$|Z(G)| = p^n$,由归纳假设$G/Z(G)$一定包含一个阶数为$p^{l-n}$的正规子群$N$,显然$N$对应了一个$G$中的正规子群,它的阶是$p^l$。


$\square$
\end{document}
